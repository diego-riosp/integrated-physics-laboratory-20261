\section*{Objetivos}

\begin{itemize}
    \item Comprender el proceso de medición y expresar correctamente el resultado de una medida.
    \item Calcular el error propagado y la incertidumbre de una medición.
    \item Entender promedios y estadísticas que se pueden realizar con medidas tomadas por los equipos de medida.
\end{itemize}

\section*{Conceptos a estudiar}

Uso adecuado de instrumentos de medida y propagación de errores.

\section*{Procedimiento}

Medir masa y volumen de objetos con distintas formas geométricas para obtener su densidad por el método geométrico, usando la relación:

\begin{equation}
\rho \pm \Delta\rho = \frac{m \pm \Delta m}{V \pm \Delta V}
\label{eq:densidad}
\end{equation}

Siendo $\rho$ la densidad del objeto, $m$ y $V$ su masa y volumen, respectivamente.

\rule{\textwidth}{1pt}

\section*{Actividad 1: Densidad}

\textbf{Materiales:} Flexómetro, pie de rey, balanza digital, balanza analógica, tornillo micrométrico, cronómetro, beaker.

\begin{itemize}
    \item Medir la masa y volumen de cada objeto reportando cada medida con su correspondiente incertidumbre (de acuerdo a la precisión del instrumento de medida).
    \item Aplicando la ecuación (\ref{eq:densidad}), determinar la densidad de los objetos. Propagar errores para reportar la incertidumbre de la densidad (ver guía de manejo de errores).
    \item Utilizar criterios de redondeo para expresar los resultados con la apropiada cantidad de cifras significativas.
    \item Si se conoce el material del objeto, determinar el error relativo porcentual.
\end{itemize}


\textbf{Preguntas orientadoras:}
\begin{itemize}
    \item ¿Qué quieren decir cada uno de los errores calculados?
    \item ¿Cómo se relacionan?
    \item ¿Cómo nos ayudan a obtener conclusiones?
\end{itemize}


\rule{\textwidth}{1pt}