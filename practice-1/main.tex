\documentclass[letterpaper]{article}
\usepackage{geometry}
\geometry{
papersize = {215mm, 330mm},
headsep = 0.2cm,
head = 3cm,
top = 1.5cm,
bottom = 2.3cm,
left = 1.3cm,
right = 1.3cm,
foot = 1.5cm,
includehead,
includefoot,
}


\usepackage{inputenc}
\usepackage[spanish, es-noshorthands]{babel}
\usepackage{textcomp}
\usepackage{lastpage}
\usepackage{framed}
\usepackage{float}
\usepackage{multirow, array}
\usepackage{amsmath}
\usepackage{amssymb}
\usepackage{amsfonts}
\usepackage{graphicx}
\usepackage{subfigure}
\usepackage{hyperref}
\usepackage{chngcntr}
\usepackage{multicol}
\title{\ttitle}
\usepackage{hyperref}
\hypersetup{urlcolor=blue, colorlinks=true}
%Extensiones admitidas:
\DeclareGraphicsExtensions{.pdf,.png,.jpg}
\newcommand{\pdfgraphics}{\DeclareGraphicsExtensions{.png,.pdf,.jpg}}

\usepackage{fancyhdr}
\pagestyle{fancy}


\newcolumntype{P}[1]{>{\centering\arraybackslash}p{#1}}
\newcolumntype{M}[1]{>{\centering\arraybackslash}m{#1}}
\newcolumntype{B}[1]{>{\centering\arraybackslash}b{#1}}


\setlength{\parindent}{0pt}
\linespread{1.2}

\def\hs{\hspace{0.1cm}}

\title{Estadística}
\author{Diego Ríos}

\usepackage[table]{xcolor}
\definecolor{black}{rgb}{0.0, 0.0, 0.0}
\renewcommand{\arrayrulewidth}{0.8pt}
\fancyhf{}
\renewcommand{\headrulewidth}{0pt}
\arrayrulecolor{black}

\chead{
\renewcommand{\arraystretch}{1.4}
\begin{tabular}{|m{1.9cm}|M{11.9cm}|m{3.7cm}|}
\hline \vspace{0.1em}
\multirow{4}{*}{\includegraphics[scale=0.045]{figures/logo.png}
} & \multirow{2}{*}{\textbf{\topic}} & {\textbf{Instituto de Física}} \\ \cline{3-3} 
                  &                                               & \textbf{Profesor:} Diego Ríos \\ \cline{2-3} 
                  & \multirow{2}{*}{\textbf{LABORATORIO INTEGRADO DE FÍSICA}}                                              & \textbf{Fecha:} \date \\ \cline{3-3} 
                  &                                       & \textbf{Página:} \thepage \, de \pageref*{LastPage} \\ \hline
\end{tabular}
}

\cfoot{\raggedleft \color{black!70}{“Yo\dots un universo de \'atomos, un \'atomo en el universo.” \\
--Feynman}
{\raggedright \begin{picture}(0,0) 
\put(-530,0){\includegraphics[scale = 0.2]{figures/CC.png}} 
\end{picture}}
}

\begin{document}
	
	\def\ltitle{Laboratorio Integrado de Física}
\def\stitle{Laboratorio Integrado de Física}
\def\presenter{Diego R\'ios}
\def\email{Grupo 7}
\def\institute{Universidad de Antioquia}
\def\lgroup{Cursos de servicio}
\def\sgroup{2026-1}
\def\type{\exposition}
	\section*{Práctica de laboratorio: Instrumentos de medida, incertidumbre y densidad}

\section*{Objetivos}

\begin{itemize}
\item Comprender el proceso físico de medición como una interacción entre el instrumento, el observador y el sistema medido.
\item Identificar la resolución, precisión y limitaciones de distintos instrumentos de medida (gramera, pie de rey y tornillo micrométrico).
\item Expresar correctamente los resultados experimentales utilizando notación científica, cifras significativas e incertidumbres.
\item Calcular incertidumbres instrumentales (de escala) e incertidumbres estadísticas.
\item Aplicar propagación de errores en magnitudes derivadas como volumen y densidad.
\item Analizar datos experimentales mediante promedios, desviaciones estándar y regresión lineal.
\end{itemize}

\section*{Conceptos a estudiar}

\begin{itemize}
\item Medición física y error experimental.
\item Resolución y precisión instrumental.
\item Incertidumbre de escala e incertidumbre estadística.
\item Propagación de incertidumbres.
\item Cifras significativas y redondeo.
\item Promedios, desviación estándar y distribuciones gaussianas.
\item Ajuste lineal y significado físico de la pendiente.
\end{itemize}

\section*{Fundamento teórico}

La densidad de un cuerpo se define como la razón entre su masa y su volumen:

\begin{equation}
\rho = \frac{m}{V}
\end{equation}

Cuando las magnitudes medidas tienen incertidumbre, el resultado debe expresarse como:

\begin{equation}
\rho \pm \Delta \rho = \frac{m \pm \Delta m}{V \pm \Delta V}
\label{eq:densidad}
\end{equation}

La incertidumbre en la densidad se obtiene mediante propagación de errores, considerando las incertidumbres instrumentales asociadas a cada medición.

\rule{\textwidth}{1pt}

\section*{Instrumentos disponibles}

\begin{itemize}
\item Gramera (balanza digital)
\item Pie de rey (calibrador vernier)
\item Tornillo micrométrico
\end{itemize}

Cada estudiante deberá seleccionar el instrumento más adecuado según la magnitud y el tamaño del objeto a medir.

\rule{\textwidth}{1pt}

\section*{Actividad 1: Determinación de densidad por método geométrico}

\textbf{Objetivo:} Determinar la densidad de distintos objetos sólidos a partir de mediciones directas de masa y dimensiones geométricas, incluyendo sus respectivas incertidumbres.

\subsection*{Objetos}

\begin{itemize}
\item Cilindro
\item Esfera
\item Tuerca
\item Cubo
\end{itemize}

\subsection*{Procedimiento}

\begin{itemize}
\item Medir la masa de cada objeto utilizando la gramera.
\item Medir las dimensiones geométricas necesarias (diámetros, longitudes, espesores) usando el pie de rey o el tornillo micrométrico, según corresponda.
\item Calcular el volumen de cada objeto usando las fórmulas geométricas apropiadas.
\item Asignar a cada medición su incertidumbre de escala de acuerdo con la resolución del instrumento.
\item Calcular la densidad de cada objeto utilizando la ecuación (\ref{eq:densidad}).
\item Propagar las incertidumbres para obtener $\Delta V$ y $\Delta \rho$.
\item Expresar todos los resultados con cifras significativas correctas y notación científica.
\item Si se conoce el material del objeto, calcular el error relativo porcentual entre el valor experimental y el valor tabulado.
\end{itemize}

\subsection*{Resultados esperados}

Para cada objeto se debe reportar:

\begin{itemize}
\item Masa: $m \pm \Delta m$
\item Volumen: $V \pm \Delta V$
\item Densidad: $\rho \pm \Delta \rho$
\end{itemize}

\rule{\textwidth}{1pt}

\section*{Actividad 2: Análisis estadístico y regresión lineal}

\textbf{Objetivo:} Estudiar la consistencia experimental de la densidad de un mismo material y su relación lineal masa--volumen.

\subsection*{Sistema}

9 cubos de madera del mismo material.

\subsection*{Procedimiento}

\begin{itemize}
\item Medir la masa y las dimensiones de cada cubo.
\item Calcular el volumen y la densidad de cada uno, incluyendo las incertidumbres de escala.
\item Calcular la densidad promedio del conjunto de cubos.
\item Construir una gráfica de masa vs volumen para los 9 cubos, incluyendo barras de error instrumentales.
\item Realizar un ajuste lineal del tipo:


\begin{equation}
m = \rho \, V
\end{equation}

\item Interpretar físicamente la pendiente de la recta como la densidad del material.
\item Comparar:
\begin{itemize}
    \item La densidad promedio calculada directamente.
    \item La densidad obtenida como pendiente del ajuste lineal.
\end{itemize}
\item Analizar la consistencia entre ambos valores dentro de sus respectivas incertidumbres.


\end{itemize}

\subsection*{Resultados esperados}

\begin{itemize}
\item Tabla de datos experimentales.
\item Gráfica $m$ vs $V$ con barras de error.
\item Ecuación de la recta ajustada.
\item Valor de la densidad por promedio y por regresión.
\item Análisis comparativo e interpretación física.
\end{itemize}

\rule{\textwidth}{1pt}

\section*{Actividad 3: Distribución estadística de la densidad (monedas de $50$ COP)}

\textbf{Objetivo:} Analizar estadísticamente la densidad de un conjunto grande de objetos idénticos y estudiar su distribución.

\subsection*{Sistema experimental}

\begin{itemize}
\item 30 monedas de $50$ COP (nueva generación) por equipo.
\item 7 equipos de trabajo.
\end{itemize}

Total de datos:

\begin{itemize}
\item 210 mediciones de masa.
\item 7 mediciones de volumen.
\end{itemize}

\subsection*{Procedimiento}

\begin{itemize}
\item Cada equipo mide la masa de sus 30 monedas (reportar \href{https://docs.google.com/spreadsheets/d/1CiCF3MJgho7KHePhqMUMBFrn_ewDAEySctYWGPo965I/edit?usp=sharing}{aquí} las mediciones).
\item Cada equipo selecciona aleatoriamente una moneda y mide su volumen mediante medición de diámetro y espesor con tornillo micrométrico.
\item Se calcula el volumen de cada una de las 7 monedas medidas.
\item Se obtiene el volumen promedio $\bar{V}$ del conjunto.
\item Se calcula la densidad de cada una de las 210 monedas usando:


\begin{equation}
\rho_i = \frac{m_i}{\bar{V}}
\end{equation}

\item Se construye la distribución de densidades del conjunto.
\item Se calcula la media y la desviación estándar de la densidad (incertidumbre estadística).
\item Se construye el histograma de densidad.
\item Se ajusta una distribución gaussiana teórica a los datos.

\end{itemize}

\subsection*{Resultados esperados}

\begin{itemize}
\item Histograma de densidad.
\item Curva gaussiana ajustada.
\item Valor medio de la densidad.
\item Desviación estándar.
\item Interpretación estadística de la dispersión de los datos.
\end{itemize}

\rule{\textwidth}{1pt}

\section*{Preguntas orientadoras generales}

\begin{itemize}
\item ¿Qué representa físicamente la incertidumbre de una medición?
\item ¿Cuál es la diferencia entre incertidumbre instrumental e incertidumbre estadística?
\item ¿Por qué diferentes métodos pueden producir valores distintos de densidad?
\item ¿Cuándo dos resultados experimentales pueden considerarse compatibles?
\item ¿Qué información física aporta la forma de una distribución gaussiana?
\end{itemize}

\rule{\textwidth}{1pt}

\section*{Actividad 4: Determinación de $g$ en plano inclinado (incertidumbre estadística y de escala)}

\textbf{Objetivo:} Profundizar en el concepto de incertidumbre estadística asociada al tiempo de reacción humano y combinarla con incertidumbres instrumentales en la determinación experimental de la aceleración gravitacional.

\subsection*{Sistema experimental}

\begin{itemize}
\item Plano inclinado con medidor de ángulo incorporado
\item Regla o cinta métrica
\item Cronómetro
\item Objeto deslizante o rodante
\end{itemize}

El ángulo del plano debe fijarse en:

\begin{equation}
\theta = 10^\circ
\end{equation}

La aceleración gravitacional se determinará a partir de la relación:

\begin{equation}
g = \frac{2 d \sin(\theta)}{t^2}
\label{eq:g}
\end{equation}

donde $d$ es la longitud del plano inclinado, $\theta$ el ángulo de inclinación y $t$ el tiempo de recorrido.

\subsection*{Mediciones}

\begin{itemize}
\item $d$: medida con regla. Incertidumbre: \textbf{incertidumbre de escala}.
\item $\theta$: dado por el medidor angular del plano. Incertidumbre: \textbf{incertidumbre de escala}.
\item $t$: medido con cronómetro. Incertidumbre: \textbf{incertidumbre estadística} (desviación estándar).
\end{itemize}

\subsection*{Diseño experimental}

Cada equipo deberá escoger una de las siguientes dos opciones:

\subsubsection*{Opción A: Variación de longitud}

\begin{itemize}
\item Seleccionar tres longitudes diferentes del plano: $d_1, d_2, d_3$.
\item Para cada longitud, medir el tiempo de recorrido del objeto \textbf{40 veces}.
\item Calcular para cada $d_i$:
\begin{itemize}
\item Tiempo promedio $\bar{t}i$
\item Desviación estándar $\sigma{t_i}$
\item Valor de $g_i$ usando la ecuación (\ref{eq:g})
\end{itemize}
\end{itemize}

\subsubsection*{Opción B: Longitud fija}

\begin{itemize}
\item Seleccionar una única longitud fija $d$.
\item Medir el tiempo de recorrido del objeto \textbf{150 veces}.
\item Calcular:
\begin{itemize}
\item Tiempo promedio $\bar{t}$
\item Desviación estándar $\sigma_t$
\item Valor de $g$ usando la ecuación (\ref{eq:g})
\end{itemize}
\end{itemize}

\subsection*{Análisis de incertidumbre}

\begin{itemize}
\item La incertidumbre en $t$ se toma como la desviación estándar de la muestra (incertidumbre estadística).
\item Las incertidumbres en $d$ y $\theta$ se toman como incertidumbres de escala.
\item La incertidumbre total en $g$ se obtiene mediante propagación de errores combinando incertidumbre estadística e instrumental.
\end{itemize}

\subsection*{Comparación con valor de referencia}

El valor experimental obtenido debe compararse con el valor de referencia local:

\begin{equation}
g_{\text{ref}} = 9.76\, \text{m/s}^2
\end{equation}

medido por la Universidad Nacional de Colombia en Medellín.

Se debe calcular el error relativo porcentual y analizar la compatibilidad del valor experimental con el valor de referencia dentro de las incertidumbres.

\subsection*{Resultados esperados}

\begin{itemize}
\item Tabla de tiempos individuales.
\item Histogramas de distribución de tiempos con fit gaussiano.
\item Valor promedio de $t$ y desviación estándar.
\item Valor de $g$ con incertidumbre total.
\item Comparación cuantitativa con $g_{\text{ref}}$.
\item Discusión sobre la influencia del tiempo de reacción en la medición.
\end{itemize}

\rule{\textwidth}{1pt}

\section*{Conclusión esperada}

Al finalizar la práctica, el estudiante deberá ser capaz de interpretar la medición como un proceso físico con límites instrumentales y estadísticos, comprender el significado real de la incertidumbre experimental y analizar datos no solo como números, sino como información física con estructura, dispersión y significado.


\end{document}