\begin{center}
\textbf{(Semana del 16 de febrero al 2 de marzo -- 3 semanas)}
\end{center}

\section*{Objetivos}

\begin{itemize}
    \item Comprender el proceso de medición y expresar correctamente el resultado de una medida.
    \item Calcular el error propagado y la incertidumbre de una medición.
    \item Entender promedios y estadísticas que se pueden realizar con medidas tomadas por los equipos de medida.
\end{itemize}

\section*{Conceptos a estudiar}

Uso adecuado de instrumentos de medida y propagación de errores.

\section*{Procedimiento}

Medir masa y volumen de objetos con distintas formas geométricas para obtener su densidad por el método geométrico, usando la relación:

\begin{equation}
\rho \pm \Delta\rho = \frac{m \pm \Delta m}{V \pm \Delta V}
\label{eq:densidad}
\end{equation}

Siendo $\rho$ la densidad del objeto, $m$ y $V$ su masa y volumen, respectivamente.

\section*{Actividad 1: Densidad}

\begin{itemize}
    \item Medir la masa y volumen de cada objeto reportando cada medida con su correspondiente incertidumbre (de acuerdo a la precisión del instrumento de medida).
    \item Aplicando la ecuación (\ref{eq:densidad}), determinar la densidad de los objetos. Propagar errores para reportar la incertidumbre de la densidad (ver guía de manejo de errores).
    \item Utilizar criterios de redondeo para expresar los resultados con la apropiada cantidad de cifras significativas.
    \item Si se conoce el material del objeto, determinar el error relativo porcentual.
\end{itemize}

\noindent
Preguntas orientadoras:
\begin{itemize}
    \item ¿Qué quieren decir cada uno de los errores calculados?
    \item ¿Cómo se relacionan?
    \item ¿Cómo nos ayudan a obtener conclusiones?
\end{itemize}

\noindent
\textbf{Nota:} Si se tienen objetos del mismo material, pero diferentes tamaños, formas o simetrías, se puede realizar un gráfico de \textbf{Masa vs Volumen} y obtener la densidad mediante un ajuste lineal al gráfico.

\section*{Actividad 2: Promedios y distribución gaussiana}

\begin{itemize}
    \item Seleccionar un objeto y medirlo alrededor de 30 veces con cada instrumento (pie de rey, micrómetro y regla).
    \item A partir de los resultados discutir los conceptos de precisión, incertidumbre, exactitud y sensibilidad.
    \item Hallar media y mediana de los datos obtenidos.
    \item Realizar gráfico de frecuencias.
\end{itemize}

\noindent
\textbf{Nota:} Otra opción puede ser que el objeto sea una moneda para hacer comparativo con el valor real dado por la página del Banco de la República y determinar el error porcentual, y la estadística de esta Actividad 2. El profesor es libre de determinar cuál prefiere trabajar con sus estudiantes.

\section*{Actividad adicional}

Si los grupos terminan antes de las tres semanas: obtener la densidad del papel por medio de la \textbf{dimensión fractal}.

\section*{Actividad 3: Dimensión fractal}

Se propone realizar el estudio de dimensión fractal, cuyo objetivo es graficar datos experimentales y aprender cómo obtener o extraer resultados relevantes de dichos gráficos.



\rule{\textwidth}{1pt}